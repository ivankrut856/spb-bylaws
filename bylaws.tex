\documentclass[12pt]{article}

\parskip=0em
\parindent=0em

\voffset=-20mm
\textheight=235mm
\hoffset=-25mm
\textwidth=180mm
\headsep=12pt
\footskip=20pt

\setcounter{page}{0}
\pagestyle{empty}

\usepackage{fancyhdr}
\lfoot{}
\cfoot{\thepage\t{/}\pageref*{LastPage}}
\rfoot{}
\renewcommand{\footrulewidth}{0.4pt}

%%%%%%%%%%%%%%%%%%%%%%%%%%%%%%%%%%%%%%%%%%%%%%%%%%%%%%%%%%%%%%%%

\usepackage{cmap}
\usepackage[T1]{fontenc}
\usepackage[utf8]{inputenc}
\usepackage[russian]{babel}

\usepackage{enumitem}

% \usepackage{lineno}
% \linenumbers

\begin{document}

\title{Устав Санкт-Петербургской Фракции Гражданского Общества}
\maketitle


Члены Гражданского Общества:

\begin{itemize}

\item руководствуясь Конституцией Российской Федерации, Федеральным законодательством Российской Федерации, иными правовыми актами, действующими в Российской Федерации;
\item руководствуясь Уставом и Манифестом Гражданского Общества;
\item выражая волю и интересы граждан Российской Федерации;
\item провозглашая незыблемость прав и личных свобод человека, принимают настоящий Устав --- основной документ Санкт-Петербургской Фракции Гражданского общества.

\end{itemize}




\renewcommand{\labelenumii}{\theenumii}
\renewcommand{\theenumii}{\theenumi.\arabic{enumii}.}

\renewcommand{\labelenumiii}{\theenumiii}
\renewcommand{\theenumiii}{\theenumii\arabic{enumiii}.}

\setlist{leftmargin=3pt}

\newcommand{\startenumerate}{\begin{enumerate}[wide, labelwidth=!, labelindent=0pt]}


\startenumerate

\item Общие положения

	\startenumerate
	\item Полное наименование – Санкт-Петербургская Фракция Гражданского Общества (далее --- Фракция).
	\item Фракция является структурным подразделением движения <<Гражданское Общество>>  (далее --- Движение).
	\item Фракция осуществляет свою деятельность на территории г. Санкт-Петербурга и Ленинградской области.
	\item Устав Фракции является основным документом, регулирующим функционирование и жизнедеятельность Фракции.  
	\item Все члены Фракции обязаны соблюдать положения настоящего Устава, Устава и Манифеста Движения, законодательства РФ. 
	\item Символикой Фракции является измененная эмблема Движения голубых цветов с линиями, схематично изображающими разведенный мост.


	\end{enumerate}

\item Цели и задачи Фракции

	\startenumerate

	\item Помимо целей и задач, закрепленных в Манифесте Движения, Фракция своими основными целями и задачами определяет:

		\startenumerate
		\item проведение и участие в свободных выборах высших должностных лиц Санкт-Петербурга и Ленинградской области, депутатов представительных органов власти;
		\item качественное улучшение городской жизни и городской среды;
		\item осуществление гражданского контроля за органами власти и местного самоуправления.
		\end{enumerate}
	
	\item Настоящий список целей и задач Фракции конкретизируется Руководящими органами Фракции.

	\end{enumerate}

\item Права и обязанности Фракции

	\startenumerate
	\item Фракция вправе:

		\startenumerate

		\item свободно распространять информацию о своей деятельности, пропагандировать свои взгляды, цели и задачи;
		\item выступать от своего имени с предложениями по различным вопросам общественной жизни, в том числе вносить соответствующие предложения в органы государственной власти и органы местного самоуправления в порядке и объеме, предусмотренном законодательством Российской Федерации;
		\item организовывать и проводить собрания, митинги, демонстрации, шествия, пикетирования, конференции, встречи, дискуссии и иные публичные мероприятия;
		\item устанавливать и поддерживать связи с политическими партиями и иными общественными объединениями, иными фракциями Движения;
		\item представлять и защищать свои права, представлять законные интересы членов и сторонников Фракции, граждан Российской Федерации в органах государственной власти, органах местного самоуправления и общественных объединениях;
		\item осуществлять добровольный сбор денежных средств с членов Фракции, третьих лиц, для достижения целей и задач, поставленных Фракцией. Размер взносов, порядок их уплаты, их введение и отмена определяется Руководящими органами Фракции;
		\item продолжить свою деятельность в случае прекращения деятельности Движения, преобразовавшись в самостоятельное политическое движение;
		\item принимать граждан Российской Федерации в члены Движения.

		\end{enumerate}

	\item Фракция обязана:

		\startenumerate
		\item соблюдать Устав Движения, добиваться выполнения пунктов Манифеста Движения;
		\item в случае добровольного сбора Фракцией денежных средств, 10\% от суммы денежных средств, полученных Фракцией, отчислять в Фонд Федерального совета Движения раз в квартал;
		\item в случае добровольного сбора Фракцией денежных средств отчитываться перед членами Фракции о сумме поступивших, потраченных денежных средств на официальных ресурсах Фракции;
		\item в течении 10 рабочих дней с даты принятия решения о принятии лица в члены Движения, информировать об этом Федеральный совет Движения.

		\end{enumerate}
 

	\end{enumerate}

\item Членство и прекращение членства во Фракции

	\startenumerate

		\item Членство во Фракции

			\startenumerate
				\item Членами Фракции могут быть только граждане Российской Федерации, достигшие восемнадцатилетнего возраста, разделяющие ее цели и задачи, соблюдающие ее Устав, принимающие участие в ее деятельности, имеющие место жительство, пребывания или нахождения на территории Санкт-Петербурга или Ленинградской области. 
				\item Членство во Фракции является добровольным и индивидуальным.
				\item Член Фракции может быть членом других фракций Гражданского общества.
				\item Для вступления во Фракцию лицо подает заявление о вступлении через официальные ресурсы Фракции.
				\item Заявление о вступлении лица в члены Фракции рассматривают Руководящие органы Фракции в течении 10 рабочих дней с даты поступления.
				\item По результатам рассмотрения заявления Руководящими органами Фракции принимается решение о приеме лица в члены Фракции, либо об отказе в приеме в члены Фракции с указанием причины отказа. В случаях, когда лицо не является членом Движения, решение о приеме лица в члены Фракции влечёт принятие в Движение и вступает в силу только после передачи информации о его принятии в Федеральный Совет Движения.
				\item О результатах рассмотрения заявления о вступлении уведомляется лицо, его подавшее.


			\end{enumerate}

		\item Прекращение членства во Фракции

			\startenumerate
				\item Лицо считается прекратившим членство во Фракции, если:
					\startenumerate
					\item лицо утратило статус члена Движения;
					\item лицо публично заявило о выходе из Фракции;
					\item лицо направило письменное заявление о выходе из Фракции в Руководящие органы Фракции;
					\item лицо было исключено из Фракции в порядке предусмотренном настоящим Уставом.
					\end{enumerate}
				\item Член Фракции может быть исключен из Фракции в случае:
					\startenumerate
					\item несоблюдения членом Фракции Устава Фракции, Манифеста или Устава Движения или иных нормативных документов Фракции или Движения.
					\item воспрепятствования членом Фракции исполнению решений Фракции;
					\item совершения членом Фракции деяний, противоречащих интересам Фракции и наносящих ей ущерб;
					\end{enumerate}
				\item Решение об исключении лица из Фракции принимается Руководящими органами Фракции.


			\end{enumerate}

	\end{enumerate}

\item Права и обязанности членов Фракции

	\startenumerate
		\item Член Фракции имеет право:

			\startenumerate
			\item избирать и быть избранным во все выборные органы Фракции;
			\item участвовать в голосовании по всем вопросам жизни Фракции;
			\item свободно излагать свои взгляды на любых мероприятиях Фракции;
			\item получать информацию о деятельности Фракции и ее органов, о деятельности структурных подразделений Фракции и их органов;
			\item обращаться с вопросами, предложениями, заявлениями в любые органы Фракции, органы ее отделений и получать ответ по существу своего обращения;
			\item участвовать в деятельности Фракции и проводимых ею мероприятиях;
			\item по поручению Руководящих органов Фракции выступать от имени Фракции;
			\item получать консультации, юридическую и иную помощь, пользоваться содействием Фракции в защите своих прав и законных интересов.

			\end{enumerate}

		\item Член Фракции обязан:

			\startenumerate

			\item соблюдать Устав Фракции, Манифест и Устав Движение и иные нормативные документы Фракции и Движения;
			\item всемерно содействовать реализации целей и задач, поставленных Фракцией;
			\item не совершать действий, дискредитирующих Фракцию, Движение в целом;
			\item исполнять решения Руководящих органов Фракции.

			\end{enumerate}


	\end{enumerate}

\item Руководящие органы Фракции

	\startenumerate

	\item Руководящим органом Фракции является Центральный совет (далее -- Совет).
	\item Совет формируется и работает составами в целом. Срок полномочий одного состава Совета составляет 2 года. Состав состоит из 5 членов.
	\item Совет принимает решения (издает нормативные документы) по всем вопросам управления Фракцией, в том числе:

	\startenumerate
		\item учреждает должности внутри фракции и принимает порядок назначения на них;
		\item распоряжается денежными средствами Фракции;
		\item принимает граждан в Движение, во Фракцию, исключает из Фракции;
		\item принимает решения о проведении демократических мероприятий;
		\item принимает иные решения, направленные на достижение целей и выполнение задач в рамках настоящего Устава.
	\end{enumerate}

	\item Решения Совета не должны противоречить настоящему Уставу, Манифесту или Уставу Движения или иным нормативным документам Движения.
	\item Решения Совета принимаются большинством голосов действующих членов Совета, если настоящим Уставом не установлено иное.

	\item Членом Совета может быть избран только член Фракции.
	\item Члены Совета избираются и доизбираются общим голосованием всех членов Фракции в рамках демократических мероприятий. Члены Совета доизбираются на срок полномочий, ограниченный сроком действующего состава. Результаты голосования утверждаются большинством голосов от принявших участие в выборном мероприятии.

	\item Совет, непосредственно после формирования, большинством голосов от состава избирает Председателя из числа членов Совета.
	\item Председатель избирается на срок полномочий действующего состава. В случае прекращения членства Председателя в Совете, Совет избирает из своего состава нового Председателя, кроме случаев, когда принимается решение о довыборах, в таких случаях Председатель избирается из доизбранного состава после проведения демократического мероприятия.
	\item Председатель назначает и проводит не реже раза в квартал заседания Совета, публично выступает от имени Фракции.

	\item Прекращение членства лица во Фракции влечет за собой прекращение его членства в Совете.
	\item В случае прекращения членства лица в Совете, Совет может принять решение о проведении довыборов. В случае, если число действующих членов Совета оказывается меньше 3, Совет принимает решение о проведении выборов или довыборов.
	\item Не позднее, чем за месяц до окончания своего срока полномочий, Совет принимает решение о проведении выборов.
	\item Полномочия Совета прекращаются с момента избрания нового состава, но не позднее 2 лет с момента формирования.

	\end{enumerate}


\item Назначение, организация и проведения демократических мероприятий

	\startenumerate

	\item Демократическими мероприятиями являются: выборы в Совет; довыборы в Совет; голосовании о правках в Устав. Они могут быть назначены Советом в порядке, предусмотренном настоящим Уставом.
	\item После назначения демократического мероприятия Советом принимается порядок проведения мероприятия, обеспечивающий возможность свободного волеизъявления всех членов Фракции и учет принявших участие в мероприятии.
	\item С момента решения о назначении демократического мероприятия до фактического проведения мероприятия должно пройти не менее месяца. Члены фракции уведомляются о дате, месте и способе проведения мероприятия по всем доступным каналам связи.
	\item Член Фракции считается принявшим участие в выборном мероприятии, если получил фактическую возможность проголосовать и был учтён как участник в соответствии с порядком проведения мероприятия.
	\item Инициативная группа, состоящая не менее, чем из половины членов Фракции, вправе подать в Совет заявление о проведении выборов в Совет до истечения срока Созыва. При получении такого заявления Совет принимает решение о проведении выборов в Совет в течение 14 суток с момента получения.
	\item Инициативная группа, состоящая не менее, чем из трети членов Фракции, вправе подать в Совет заявление о проведении голосования о правках в Устав. При получении такого заявления Совет принимает решение о проведении голосования о правках в Устав в течение 14 суток с момента получения.
	\item В случае уклонения Совета от назначения, организации или проведения выборов в Совет или голосования о правках в Устав, когда Устав предписывает эти действия, инициативная группа вправе взять эти функции на себя.

	\end{enumerate}

\item Порядок изменения Устава

	\startenumerate

	\item Обязательным условием принятия правки является поддержка правки на голосовании о правках в Устав более чем половиной членов Фракции.
	\item Заявления о проведении голосования о правках в Устав, подающееся инициативной группой, должно содержать полный список правок, предлагаемых к принятию. Решение Совета о проведении голосования о правках в Устав должно содержать полный список правок, предлагаемых к принятию, а в случае, если это решение принимается по результатам получения заявления инициативной группы, содержать все правки, содержащиеся в заявлении, и только их.
	\item Все правки, содержащиеся в заявлении инициативной группы о проведении голосования о правках в Устав или решении Совета о проведении голосования о правках в Устав, выносятся на голосование и только они. Каждая правка голосуется отдельно.
	\item Правка считается принятой после принятия решения о подтверждении правки в Устав Советом, либо по прошествии 3 дней после окончания голосования по правке, если в этот срок не было принято решение об отклонении правки в Устав.
	\item Совет может принять решение об отклонении правки в Устав в случаях, когда за правку проголосовало менее двух третей от членов Фракции. Это решение должно быть принято не менее чем 4 голосами.
	% \item Изменения в настоящий Устав вносятся по предложению и решению Ивана Шершнева.

	\end{enumerate}


\item Первый состав Совета \label{first_leader}

	\startenumerate

	\item Первый состав Совета формируется из учредителей Фракции.

	\item Члены первого состав Совета не могут быть лишены своей должности в течение двух лет, кроме как при добровольном выходе из Совета.

	\item По истечении срока полномочий первого состава Совета пункт \ref{first_leader} исключается из настоящего Устава. Исключение или изменение пункта или подпунктов иными способами не допускается.

	\end{enumerate}

\end{enumerate}

\end{document}
